\section{Fazit \& Ausblick}

\subsection{Fazit}

Bei Betrachtung der Angriffsmöglichkeiten und der Risiken wird offensichtlich, dass es nur mit beachtlichem Aufwand möglich ist, einen Webshop vollständig vor Angreifern zu schützen. Speziell aufgrund der Tatsache, dass hier nur eine Auswahl der Bedrohungen besprochen wurde.

Andererseits ist auch zu erkennen, dass viele Angriffsszenarien durch die Auswahl entsprechender Komponenten und Best--Practices wirkungsvoll abgewehrt werden können. Wenn Sicherheitsaspekte schon von Beginn an in den Softwareentwurf eingehen, ist die resultierende Applikation zumindest vor den meisten Angriffen geschüzt. Zumindest zufälligen Angreifern, die im Internet nach offensichtlichen Sicherheitslücken suchen, wird somit kein Ziel geboten.

\subsection{Ausblick}

Auf dem Weg zu einer tatsächlichen Implementierung ist es notwendig, weitere Angriffsmethoden zu betrachten, damit das fertige Produkt zumindest gegen die von der OWASP erstellten Top--10 Liste bestehen kann. Die jeweils notwendigen Gegenmaßnahmen sollten als Richtlinien für die Codierung festgelegt und überprüft werden. 

Falls der Internetshop um eine Bezahlfunktion erweitert werden soll, sind noch weitere Angriffsziele zu beachten, da in diesem Fall nicht nur Daten, sondern auch Geld\footnote{Das Geld unserer Kunden} sowie Waren\footnote{Die angebotenen Waren ohne Bezahlung} erbeutet werden können. Dieses Szenario würde aber den Umfang dieser Arbeit überschreiten und müsste somit seperat erörtert werden.

Für weitere Untersuchungen würde sich die Fragestellungen anbieten, in wie weit die am Markt erhältliche Shopsoftware gegen Angriffe geschützt ist bzw. wie groß der Entwicklungsaufwand ist, den am Markt üblichen Grad an Sicherheit bei einer selbst erstellten Anwendung zu erreichen. Hierdurch könnte eine Make-or-Buy Entscheidung unterstützt bzw. begründet werden.